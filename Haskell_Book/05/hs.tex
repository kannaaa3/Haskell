\documentclass{article}
\usepackage[utf8]{inputenc}
\usepackage{amsmath, amssymb, systeme, mathtools, lmodern, float, graphicx, fontawesome5, titlesec, multicol}
\usepackage[dvipsnames]{xcolor}
\usepackage[most]{tcolorbox}
\usepackage[scale=.95,type1]{cabin}
\usepackage[framemethod=tikz]{mdframed}

\usepackage[legalpaper,margin=1in]{geometry}

\setlength{\parindent}{10pt}
% \setlength{\parskip}{1em}
\renewcommand{\baselinestretch}{1.2}

\title{Basic Datatypes}
\date{}

\newcommand\y{\cellcolor{blue!10}}

\usepackage{tabularray}
\SetTblrInner{colsep=5pt,rowsep=1pt}

\newcounter{Def}[section]
\newenvironment{Def}[1][]{%
  \ifstrempty{#1}%
  {\mdfsetup{%
    frametitle={%
      \tikz[baseline=(current bounding box.east),outer sep=0pt]
      \node[line width=1pt,anchor=east,rectangle,draw=blue!20,fill=white]
    {\strut \color{black}{Definition}~};}}
  }%
  {\mdfsetup{%
    frametitle={%
      \tikz[baseline=(current bounding box.east),outer sep=0pt]
      \node[line width=1pt,anchor=east,rectangle,draw=Lavender!20,fill=white]
    {\strut ~\color{RubineRed!80}{#1}};}}%
  }%
  \mdfsetup{innertopmargin=2pt,linecolor=Lavender!20,%
            linewidth=1pt,topline=true,%
            frametitleaboveskip=\dimexpr-\ht\strutbox\relax,}
  \begin{mdframed}[]\relax%
  }{\end{mdframed}}
\titleformat{\section}
  {\fontfamily{lmss}\selectfont\LARGE\bfseries\color{black}}
  {\thesection}{1em}{}
\begin{document}
    \subsubsection*{Partial application}
    \begin{multicols}{2}
      \begin{itemize} \renewcommand\labelitemi{\small \textcolor{Lavender}{$\blacksquare$}}
        \item {\fontfamily{cmtt}\selectfont Uncurried functions.} 1 function, many arguments.
        \item {\fontfamily{cmtt}\selectfont Curried functions.} Many functions, 1 argument apiece.
      \end{itemize}
   \end{multicols} 

\begin{minipage}[]{0.4\linewidth}
  \begin{Def}[Typecheck]
 You can check types that \textit{aren't implemented yet}. Give it an \textit{undefined} to bind the signature.
  \end{Def}
\end{minipage}\hfill
\begin{minipage}[]{0.55\linewidth}
  3 types of type signatures: concrete, constrained polymorphic, and parametrically polymorphic.
\end{minipage}

$ \ $

        \textbf{Parametricity} means that the behavior of a function with respect to the types of its (parametrically polymorphic) arguments is uniform. The behavior cannot change just because it was applied to
        an argument of a different type.

    \section{Type inference}
    Don't have to assert a type for value because Haskell has \textbf{type inference} (algorithm); which 
        infer the most generally applicable (polymorphic) type that is still correct.

    \begin{Def}[Monomorphism restriction]
        Top-level declarations by default will have a concrete type if any can be determined.
        
        \includegraphics*[width=11cm]{mono.png}
    \end{Def}

    \section{Definition}
    \begin{multicols}{2}
    \begin{enumerate}
        \item \textbf{Polymorphism}:  \textit{parametric} or \textit{ad-hoc polymorphism}. Smaller sets from a large set.
        \item \textbf{Type inference:} Infer principal types from terms without needing explicit type
            annotations.
        \item \textbf{Type variable} is: {\fontfamily{cmtt}\selectfont a} $\to$ {\fontfamily{cmtt}\selectfont a}
        \item \textbf{Typeclass:} a means of expressing faculties or interfaces that multiple datatypes may have
            in common.
        \item \textbf{Ad-hoc polymorphism} (\textit{constrained polymorphism})
        \item \textbf{Module}: The unit of organization
    \end{enumerate}
    \end{multicols}




\end{document}

