\documentclass{article} 
\usepackage[utf8]{inputenc}
\usepackage{amsmath, amssymb, systeme, mathtools, lmodern, float, graphicx, listings, titlesec}
\usepackage[dvipsnames]{xcolor}
\usepackage[scale=.95,type1]{cabin}
\usepackage[framemethod=tikz]{mdframed}

\usepackage[legalpaper,margin=1in]{geometry}

\setlength{\parindent}{10pt}
% \setlength{\parskip}{1em}
\renewcommand{\baselinestretch}{1.2}

\title{CHAPTER 7: MORE FUNCTIONAL PATTERNS}
\date{}
\author{}

\newcounter{Def}[section]
\newenvironment{Def}[1][]{%
  \ifstrempty{#1}%
  {\mdfsetup{%
    }
  }%
  {\mdfsetup{%
    frametitle={%
      \tikz[baseline=(current bounding box.east),outer sep=0pt]
      \node[line width=1pt,anchor=east,rectangle,draw=Lavender!20,fill=white]
    {\strut \color{RubineRed!80}{#1}};}}%
  }%
  \mdfsetup{innertopmargin=2pt,linecolor=Lavender!20,%
            linewidth=1pt,topline=true,%
            frametitleaboveskip=\dimexpr-\ht\strutbox\relax,}
  \begin{mdframed}[]\relax%
  }{\end{mdframed}}

\definecolor{codegreen}{rgb}{0,0.6,0}
\definecolor{codegray}{rgb}{0.5,0.5,0.5}
\definecolor{backcolour}{rgb}{0.93725490196,0.94509803921,0.96078431372}
\definecolor{codewhite}{rgb}{0.75,0.78,0.84}
\definecolor{normalcode}{rgb}{0.35,0.36,0.45}

\lstdefinestyle{mystyle}{
    backgroundcolor=\color{backcolour},   
    commentstyle=\color{codegray},
    keywordstyle=\color{magenta},
    numberstyle=\small\color{Gray!70}\fontfamily{cmtt}\selectfont,
    stringstyle=\color{codegreen},
    basicstyle=\ttfamily\color{normalcode}\footnotesize,
    breakatwhitespace=false,         
    % frame=single,
    % breaklines=true,                 
    captionpos=b,                    
    keepspaces=true,                 
    numbers=left,                    
    numbersep=5pt,                  
    showspaces=false,                
    showstringspaces=false,
    showtabs=false,                  
    tabsize=2
}

\lstset{style=mystyle}

\titleformat{\section}
  {\fontfamily{lmss}\selectfont\LARGE\bfseries\color{black}}
  {\thesection}{1em}{}
\begin{document}
  \maketitle
    A value that can be used as an argument to a function is a \textit{first-class} value.

    \subsubsection*{Some ERROR cases!!!}
   The first one, not in scope because $y$ was \textit{inside} the {\fontfamily{cmtt}\selectfont let} expression. 
    \begin{lstlisting}[language = Haskell]
bindExp :: Integer -> String
bindExp x = let z = y + x in -- Here y is not in scope!!!
            let y = 5 in "the integer was: "
            ++ show x ++ " and y was: "
            ++ show y ++ " and z was: " ++ show z \end{lstlisting}
This time, {\fontfamily{cmtt}\selectfont x = 10} shadows the {\fontfamily{cmtt}\selectfont x} argument.
            \begin{lstlisting}[language = Haskell]
bindExp :: Integer -> String
bindExp x = let x = 10; y = 5 in
            "the integer was: " ++ show x
            ++ " and y was: " ++ show y 
\end{lstlisting}
    The reason is that Haskell has \textbf{lexical/static scoping} (depends on the location in the code and the lexical context - like {\fontfamily{cmtt}\selectfont let} or {\fontfamily{cmtt}\selectfont where} clauses).

\subsubsection*{Anonymous functions}
\begin{minipage}[]{0.3\linewidth}
\begin{lstlisting}
Prelude> (\x -> x * 3) 5
15
\end{lstlisting}    
  \end{minipage}\hfill
\begin{minipage}[]{0.65\linewidth}
 It doesn't need a name, since it's called only once.
\end{minipage}


\begin{minipage}[]{0.65\linewidth}
\subsubsection*{Pattern matching}

If {\fontfamily{cmtt}\selectfont x} is not matched yet, {\fontfamily{cmtt}\selectfont f x = } \emph{bottom} (a non-value - the program can't return a value/result - like infinite loop) and throw an exception.

    \end{minipage}\hfill
    \begin{minipage}[]{0.3\linewidth}
\begin{lstlisting}
 *** Exception: :50:33-48:
   Non-exhaustive patterns
       in function isItTwo
\end{lstlisting}
    \end{minipage}

    \subsubsection*{Pattern matching against data constructors}

    Some data constructors have \textbf{parameters}, and pattern matching helps expose
    the data in their arguments.
    \begin{itemize} \renewcommand\labelitemi{\small \textcolor{Lavender}{$\blacksquare$}}
      \item {\fontfamily{cmtt}\selectfont newtype} is {\fontfamily{cmtt}\selectfont data} but only 1 field and 1 constructor. 
\begin{lstlisting}[language = Haskell]
newtype Username = Username String -- 1 field and 1 constructor \end{lstlisting} 
    \end{itemize}

    \begin{Def}[Higher-order function]
        \textit{Higher-order functions} are functions that accept functions as arguments. Functions are 
        values.
    \end{Def}
    
    Guards evaluate \textbf{sequentially}, order it from the most common case to the least.

    \subsection*{Function composition and "pointfree" style}
\begin{lstlisting}[language = Haskell]
(f . g) x = f (g x)
Prelude> negate . sum $ [1, 2, 3, 4, 5] -- or (negate . sum) [1, 2, 3, 4, 5]
-15 
Prelude> let f x = take 5 . enumFrom $ x
Prelude> f 3
[3,4,5,6,7]
\end{lstlisting}

Btw, using {\fontfamily{cmtt}\selectfont \$} makes the \textit{applications} happen \textbf{after} functions are composed.

    \begin{Def}[Currying]
        \textit{Currying} is the process of transforming a function that takes multiple arguments into
        \textbf{a series of functions} which each take 1 argument and return one result.
    \end{Def}

    And don't use error.

    
    
\end{document}
