\documentclass{article}
\usepackage[utf8]{inputenc}
\usepackage{amsmath, amssymb, systeme, mathtools, lmodern, float, graphicx, titlesec}
\usepackage[dvipsnames]{xcolor}
\usepackage[scale=.95,type1]{cabin}
\usepackage[framemethod=tikz]{mdframed}

\usepackage[legalpaper,margin=1in]{geometry}

\setlength{\parindent}{10pt}
% \setlength{\parskip}{1em}
\renewcommand{\baselinestretch}{1.2}

\title{Hello, Haskell!}
\author{}
\date{}

\usepackage{tabularray}

\newcounter{Def}[section]
\newenvironment{Def}[1][]{%
  \ifstrempty{#1}%
  {\mdfsetup{%
    frametitle={%
      \tikz[baseline=(current bounding box.east),outer sep=0pt]
      \node[line width=1pt,anchor=east,rectangle,draw=blue!20,fill=white]
    {\strut \color{black}{Definition}~};}}
  }%
  {\mdfsetup{%
    frametitle={%
      \tikz[baseline=(current bounding box.east),outer sep=0pt]
      \node[line width=1pt,anchor=east,rectangle,draw=blue!20,fill=white]
    {\strut \color{black}{Definition}~:~\color{blue4}{#1}};}}%
  }%
  \mdfsetup{innertopmargin=2pt,linecolor=blue!20,%
            linewidth=1pt,topline=true,%
            frametitleaboveskip=\dimexpr-\ht\strutbox\relax,}
  \begin{mdframed}[]\relax%
  }{\end{mdframed}}

\titleformat{\section}
  {\fontfamily{lmss}\selectfont\LARGE\bfseries\color{black}}
  {\thesection}{1em}{}
\begin{document}
\maketitle
   Redexes are reducible expressions. (Also called "\textit{normalizing}" or "executing" an expression)
   
   Functions are a specific type of expression. Currying is applying a series of nested 
   functions.

   \section{Evaluation}
   Evaluating an expression is reducing the terms until the expression reaaches its simplest form.
   \section{Associativity and precedence}
   Higher number, higher precedence.
   Associativity: left or right.

   Use space, not tab.

   Error: indent, not starting a declaration at the beggining (left) column of the line.
   
   The first declaration in a module defines the remaining ones.

   (quot x y) * y + (rem x y) == x
   (div x y) * y + (mod x y) == x

   mod has the same sign as the divisor.
   rem: dividend

   sectioning (+1)


   \section{Let and where}
   Let introduces an expression, where is a declaration.

   Scope is the area of source code where a binding of a variable applies.
    
   Write code in a source file: the order is unimportant, but when writing code directly into
   the REPL, it does matter.

   Infix: place between.

   \begin{Def}
       \textbf{Syntactic sugar} is syntax within a programming language designed to make expressions easier
       to write or read.
   \end{Def}

   Types are way of categorizing values.

   :: has the type (type signature)

   \begin{Def}[Typeclasses]
       Typeclasses provide definitions of operations, or functions, that can be shared
       across sets of types.
   \end{Def}
   (:) cons
    
   \begin{Def}[Scope]
       Scope is where a variable referred to by name is valid. Another word used with the same
       meaning is \textit{visibility}, because if a variable isn't visible, it's not in scope
   \end{Def}

   \begin{Def}[Local and Top level bindings]
       \begin{itemize}
           \item Local bindings are bindings local to particular expressions. Cannot be imported
       by other programs or modules.
   \item Top level bindings are bindings that stand outside of any other declaration. Can be made available
       to other modules within your programs or to ther ppl.
       \end{itemize}
   \end{Def}

   \begin{Def}[Data Structures]
       Data structures are way of organizing data so that the data can be accessed conveniently
       or efficiently.
   \end{Def}







\end{document}
