\documentclass{article} 
\usepackage[utf8]{inputenc}
\usepackage{amsmath, amssymb, systeme, mathtools, lmodern, float, graphicx}
\usepackage[most]{tcolorbox}
\usepackage[scale=.95,type1]{cabin}
\usepackage[framemethod=tikz]{mdframed}


\usepackage[legalpaper,margin=1in]{geometry}

\setlength{\parindent}{10pt}
\setlength{\parskip}{1em}
\renewcommand{\baselinestretch}{1.2}

\title{Monoids and semigroups}
\date{}

\makeatletter
\renewcommand*\env@matrix[1][*\c@MaxMatrixCols c]{%
  \hskip -\arraycolsep
  \let\@ifnextchar\new@ifnextchar
  \array{#1}}
\makeatother

\newcommand\y{\cellcolor{blue!10}}

\usepackage{tabularray}
\SetTblrInner{colsep=5pt,rowsep=1pt}

\newcounter{Def}[section]
\newenvironment{Def}[1][]{%
  \ifstrempty{#1}%
  {\mdfsetup{%
    frametitle={%
      \tikz[baseline=(current bounding box.east),outer sep=0pt]
      \node[line width=1pt,anchor=east,rectangle,draw=blue!20,fill=white]
    {\strut \color{black}{Definition}~};}}
  }%
  {\mdfsetup{%
    frametitle={%
      \tikz[baseline=(current bounding box.east),outer sep=0pt]
      \node[line width=1pt,anchor=east,rectangle,draw=blue!20,fill=white]
    {\strut \color{black}{Definition}~:~\color{blue4}{#1}};}}%
  }%
  \mdfsetup{innertopmargin=2pt,linecolor=blue!20,%
            linewidth=1pt,topline=true,%
            frametitleaboveskip=\dimexpr-\ht\strutbox\relax,}
  \begin{mdframed}[]\relax%
  }{\end{mdframed}}
%{\fontfamily{cmtt}\selectfont }

\begin{document}
 
  \begin{Def}[Monoid]
    A monoid is a binary associative operation with an identity. Follows 2 laws: associativity
    and identity.
  \end{Def}
  Integer form a monoid under summation and multiplication. Lists form a monoid under concatenation.

  \subsection*{Orphan instances}
  Writing orphan instances should be avoided \textit{at all costs}.

  An orphan instance is when an instance is defined for a datatype and typeclass, but not
  in the same module as either the declaration of the typeclass or the datatype. If not own
  a typeclass, newtype it!

  \begin{enumerate}
    \item Put the instance in the same module as the type.
    \item Put the instance in the same module as the typeclass definition.
    \item Neither the type nor the typeclass are our, then newtype.
  \end{enumerate}

  \begin{Def}[Semigroup]

{\fontfamily{cmtt}\selectfont Semigroup} provides a binary associative operation.
  \end{Def}





\end{document}
