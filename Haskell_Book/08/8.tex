\documentclass{article} 
\usepackage[utf8]{inputenc}
\usepackage{amsmath, amssymb, systeme, mathtools, lmodern, float, graphicx, listings, titlesec, fontawesome5}
\usepackage[dvipsnames]{xcolor}
\usepackage[scale=.95,type1]{cabin}
\usepackage[framemethod=tikz]{mdframed}

\usepackage[legalpaper,margin=1in]{geometry}

\setlength{\parindent}{10pt}
% \setlength{\parskip}{1em}
\renewcommand{\baselinestretch}{1.2}

\title{CHAPTER 8: RECURSIONS}
\date{}
\author{}

\newcounter{Def}[section]
\newenvironment{Def}[1][]{%
  \ifstrempty{#1}%
  {\mdfsetup{%
    }
  }%
  {\mdfsetup{%
    frametitle={%
      \tikz[baseline=(current bounding box.east),outer sep=0pt]
      \node[line width=1pt,anchor=east,rectangle,draw=Lavender!20,fill=white]
    {\strut \color{RubineRed!80}{#1}};}}%
  }%
  \mdfsetup{innertopmargin=2pt,linecolor=Lavender!20,%
            linewidth=1pt,topline=true,%
            frametitleaboveskip=\dimexpr-\ht\strutbox\relax,}
  \begin{mdframed}[]\relax%
  }{\end{mdframed}}

\definecolor{codegreen}{rgb}{0,0.6,0}
\definecolor{codegray}{rgb}{0.5,0.5,0.5}
\definecolor{backcolour}{rgb}{0.93725490196,0.94509803921,0.96078431372}
\definecolor{codewhite}{rgb}{0.75,0.78,0.84}
\definecolor{normalcode}{rgb}{0.35,0.36,0.45}

\lstdefinestyle{mystyle}{
    backgroundcolor=\color{backcolour},   
    commentstyle=\color{codegray},
    keywordstyle=\color{magenta},
    numberstyle=\small\color{Gray!70}\fontfamily{cmtt}\selectfont,
    stringstyle=\color{codegreen},
    basicstyle=\ttfamily\color{normalcode}\footnotesize,
    breakatwhitespace=false,         
    % frame=single,
    % breaklines=true,                 
    captionpos=b,                    
    keepspaces=true,                 
    numbers=left,                    
    numbersep=5pt,                  
    showspaces=false,                
    showstringspaces=false,
    showtabs=false,                  
    tabsize=2
}

\lstset{style=mystyle}

\titleformat{\section}
  {\fontfamily{lmss}\selectfont\LARGE\bfseries\color{black}}
  {\thesection}{1em}{}
\begin{document}
\maketitle
\subsubsection*{Recursions}
  Basically, recursion is self referential composition.

\begin{lstlisting}[language = Haskell]
module Factorial where

factorial :: Integer -> Integer
factorial 0 = 1
factorial n = n * factorial (n - 1)

-- explaination
brokenFact1 4 =
  4 * (4 - 1)
  * ((4 - 1) - 1)
  * (((4 - 1) - 1) - 1)
  * ((((4 - 1) - 1) - 1) - 1) \end{lstlisting}


    \begin{Def}[Bottom]
        \textit{Bottom} or $\perp$ is a term used in Haskell to refer to computations that do not
        successfully result in value.

  \begin{lstlisting}[language = Haskell]
f :: Bool -> Int
f False = 0
f _ = error $ "*** Exception: "
            ++ "Non-exhaustive"
            ++ "patterns in function f" \end{lstlisting}
    \end{Def}
\subsubsection*{\textcolor{Purple}{ {\small \faIcon{wrench}} ERRORS!!!}}

  I had a problem with \textbf{function composition} and \textbf{recursion}. Now it's pretty hard to know how it works when it comes to more compositions.
\begin{lstlisting}[language = Haskell]
digits :: Int -> [Int]
digits = reverse . digits'
  where digits' n
          | n < 10    = [n]
          | otherwise = lastDigit : digits nReduced
            where divmod' = n `divMod` 10
                  lastDigit = snd divmod'
                  nReduced  = fst divmod' \end{lstlisting}
  or even this
                  \begin{lstlisting}[language = Haskell]
digits :: Int -> [Int]
digits n = 
  let y = digits' n
  in reverse y
    where digits' n
            | n < 10    = [n]
            | otherwise = lastDigit : digits nReduced
              where divmod' = n `divMod` 10
                    lastDigit = snd divmod'
                    nReduced  = fst divmod' \end{lstlisting}
The result: 
                    \begin{lstlisting}[language = Haskell]
Prelude> digits 1234567
[6,4,2,1,3,5,7]\end{lstlisting}


\end{document}
