\documentclass{article} 
\usepackage[utf8]{inputenc}
\usepackage{amsmath, amssymb, systeme, mathtools, lmodern, float, graphicx, listings, titlesec, fontawesome5}
\usepackage[dvipsnames]{xcolor}
\usepackage[scale=.95,type1]{cabin}
\usepackage[framemethod=tikz]{mdframed}

\usepackage[legalpaper,margin=1in]{geometry}

\setlength{\parindent}{10pt}
% \setlength{\parskip}{1em}
\renewcommand{\baselinestretch}{1.2}

\title{CHAPTER 9: LISTS}
\date{}
\author{}

\newcounter{Def}[section]
\newenvironment{Def}[1][]{%
  \ifstrempty{#1}%
  {\mdfsetup{%
    }
  }%
  {\mdfsetup{%
    frametitle={%
      \tikz[baseline=(current bounding box.east),outer sep=0pt]
      \node[line width=1pt,anchor=east,rectangle,draw=Lavender!20,fill=white]
    {\strut \color{RubineRed!80}{#1}};}}%
  }%
  \mdfsetup{innertopmargin=2pt,linecolor=Lavender!20,%
            linewidth=1pt,topline=true,%
            frametitleaboveskip=\dimexpr-\ht\strutbox\relax,}
  \begin{mdframed}[]\relax%
  }{\end{mdframed}}

\definecolor{codegreen}{rgb}{0,0.6,0}
\definecolor{codegray}{rgb}{0.5,0.5,0.5}
\definecolor{backcolour}{rgb}{0.93725490196,0.94509803921,0.96078431372}
\definecolor{codewhite}{rgb}{0.75,0.78,0.84}
\definecolor{normalcode}{rgb}{0.35,0.36,0.45}

\lstdefinestyle{mystyle}{
    backgroundcolor=\color{backcolour},   
    commentstyle=\color{codegray},
    keywordstyle=\color{magenta},
    numberstyle=\small\color{Gray!70}\fontfamily{cmtt}\selectfont,
    stringstyle=\color{codegreen},
    basicstyle=\ttfamily\color{normalcode}\footnotesize,
    breakatwhitespace=false,         
    % frame=single,
    % breaklines=true,                 
    captionpos=b,                    
    keepspaces=true,                 
    numbers=left,                    
    numbersep=5pt,                  
    showspaces=false,                
    showstringspaces=false,
    showtabs=false,                  
    tabsize=2
}

\lstset{style=mystyle}

\titleformat{\section}
  {\fontfamily{lmss}\selectfont\LARGE\bfseries\color{black}}
  {\thesection}{1em}{}
\begin{document}
\begin{lstlisting}[language = Haskell]
 [x ** 2 | x <- [1..10]]  
\end{lstlisting}
   The pipe here designates the separation between  the \textbf{\textcolor{BlueGreen}{output}} function
   and the \textcolor{VioletRed}{\textbf{input}}.
\subsubsection*{Funcy Lists}
\begin{minipage}[t]{0.45\linewidth}
\begin{lstlisting}[language = Haskell]
 Prelude> [1..10]
[1,2,3,4,5,6,7,8,9,10]
Prelude> enumFromTo 1 10
[1,2,3,4,5,6,7,8,9,10]
Prelude> [2,4..10]
[2,4,6,8,10]
Prelude> enumFromThenTo 2 4 10
[2,4,6,8,10]
\end{lstlisting}
\end{minipage}\hfill
  \begin{minipage}[t]{0.45\linewidth}
\begin{lstlisting}[language = Haskell]
take :: Int -> [a] -> [a]
drop :: Int -> [a] -> [a]
splitAt :: Int -> [a] -> ([a], [a])
takeWhile :: (a -> Bool) -> [a] -> [a]
dropWhile :: (a -> Bool) -> [a] -> [a]
\end{lstlisting}
\end{minipage}

\subsubsection*{List comprehensions : Generating new list from a list, with predicates}

\begin{lstlisting}[language = Haskell]
 [x^2 | x <- [1..10], odd x] 
 [x^y | x <- [1..5], y <- [2, 3], x^y < 200] -- [1,1,4,8,9,27,16,64,25,125]
 [(x, y) | x <- [1..5], y <- [6, 7]] -- [(1,6),(1,7),(2,6),(2,7),(3,6),(3,7)]
\end{lstlisting}

\section{Spines and nonstrict evaluation}
\begin{lstlisting}[language = Haskell]
   :                     
  / \  
 1   :          -- The list 1 : (2 : (3 : [])) can be visualized like this
    / \          -- 
   2   :          --
      / \
     3  []
\end{lstlisting}        
Evaluation proceeds \textbf{up} the list. And spines are evaluated \textbf{independently} of values. Some {\fontfamily{cmtt}\selectfont func} just evaluate the spine, not the value (like {\fontfamily{cmtt}\selectfont length, ...}), but if the spine itself is \textbf{bottom}.
\begin{lstlisting}[language = Haskell]
Prelude> let x = [1] ++ undefined ++ [3]
Prelude> x
[1*** Exception: Prelude.undefined
Prelude> length x
*** Exception: Prelude.undefined
-- map is nonstrict
Prelude> take 2 $ map (+1) [1, 2, undefined]
[2,3]
In the final example, the undefined value wa
\end{lstlisting}



    \begin{Def}[WHNF: Weak Head Normal Form]
        "Normal form": the expression is fully evaluated. "Weak head normal form": the expression
        is only evaluated as far as is necessary to reach a \textbf{data constructor}.

        An expression cannot be in normal form or weak head normal form
        if the outermost part of the expression isn't a \textit{data constructor}.
        % Eg: (1, (\x -> x + 2) 2) -> (1, 2 + 2)
    \end{Def}

    \begin{lstlisting}[language = Haskell]
 Prelude> zipWith (+) [1, 2, 3] [10, 11, 12]
[11,13,15] \end{lstlisting}
    
    % \section{Product type and Sum type}
    % 
    % \begin{Def}[Product Type]
    %     A product type is type made of a set of type compounded over each other. (Tuple, data
    %     constructors with more than one argument)
    % \end{Def}
    %
    % \begin{Def}[Sum Type]
    %     A sum type is a type whose terms are terms in either type, but not simultaneously.
    % \end{Def}

    \textit{Cons cell} is a data constructor and a product of the types a and $[a]$.



\end{document}

